
\begin{DoxyRefList}
\item[Member \mbox{\hyperlink{_projekt__semestralny__drzewo__decyzyjne__sekcja06_8cpp_a0ddf1224851353fc92bfbff6f499fa97}{main}} (int argc, char $\ast$argv\mbox{[}\mbox{]})]\label{todo__todo000001}%
\Hypertarget{todo__todo000001}%
Czytanie parametrow drzewa decyzyjnego i przechowanie ich. Mozna w sumie uzyc struktury do tego by pozniej bylo latwiej z if-\/ami moze --- DONE

\label{todo__todo000002}%
\Hypertarget{todo__todo000002}%
Porownywanie wartosci do drzewa decyzyjnego -\/ tu by w sumie przydalo sie to parami sprawdzac (tzn. wyskok\mbox{[}0\mbox{]} i wzrost\mbox{[}0\mbox{]}) --- DONE (gownianie ale jest)

\label{todo__todo000003}%
\Hypertarget{todo__todo000003}%
Opracowac jak wywolac ten program z konsoli i wraz z podaniem parametrow (plikow tekstowych) -\/ ja uzywalem cl /\+EHsc Projekt\+\_\+semestralny\+\_\+drzewo\+\_\+decyzyjne.\+cpp i potem .\textbackslash{}\+Projekt\+\_\+semestralny(...) --- DONE?

\label{todo__todo000004}%
\Hypertarget{todo__todo000004}%
Opracowac zeby program podawal instrukcje co jak zrobic gdy uzytkownik nie poda plikow wejsciowych a takze moze niech poda lokalizacje gdzie zapisal plik tekstowy z wynikiem --- DONE

\label{todo__todo000005}%
\Hypertarget{todo__todo000005}%
Opisac wszystko z pomoca Doxygena -\/ ogarnac w ogole jak sie to cos robi --- DONE

\label{todo__todo000006}%
\Hypertarget{todo__todo000006}%
Have fun \+:)
\end{DoxyRefList}